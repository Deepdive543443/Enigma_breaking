\chapter{Introduction}

Enigma is one of the most important cipher machines of the early to mid of 20th century. It was used to encrypt and protect the communication between commercial, diplomatic, and most importantly, the military parties. It was adopted by Nazi German military to secure their communication. Although some larger and more complex ciphers were introduced in WWII, Enigma was still heavily adopted because of it s compact and portable.

Enigma represented a peak of classical cryptography. The breaking of Enigma and several other cipher machines might have pushed the war to end years earlier. Enigma cannot satisfy modern encryption’s requirements, but it is still an area ofstudy and research in maths and cryptography. Researchers are trying to apply modern techniques, including the trending machine learning algorithms on the task of attacking classical ciphers like Enigma. 

With the modern computers running faster and faster, machine learning has achieved great strides in the last few decades. In 1997, Hochreiter et al \cite{hochreiter1997long} presented the Long Short-Term Memory. It is still one of the most important networks in the natural language processing and time series analysis works. Deep learning made a milestone in Computer vision area in 2012 by proving deep convolutional neural networks' abilities on classifying images \cite{krizhevsky2012imagenet}. The Transformer \cite{vaswani2017attention} was presented in 2017 for machine translation tasks, and was soon extended to computer vision and multimodal applications. 

In this project, following a few researching on machine learning analyzing Enigma, we made our attempts on applied modern machine learning model on the attacking of Enigma. We implemented our network using architectures that are adopted in large language model and time series analysis (LSTM and Transformer). We used a Python Enigma API to generate our training data. Using these data, we completed a series of experiments on training our model to attack the Enigma machine. The researching code is available at: 
\url{https://github.com/Deepdive543443/Enigma_breaking_RNN_FYP/releases/tag/v1.0.2}


\section{Aims and Objectives}

This project aims to find a network architecture that could fit our requirement to attack Enigma, under evaluate its performance for that task. The trained network should be able recover unknown settings of Enigma. The network architecture would likely be architecture that is commonly used in natural language processing and time series analysis tasks. Our method of attack is analysis unknown setting by looking at the pair of ciphertext and plaintext. We plan to start by attacking one or two of Enigma’s settings, then slowly increasing the complexity of the task by training the network to attack more settings.

\section{Overview of the Report}

% This report contains four chapters. The first chapter is the Literature review of our project. In this part, we reviewed the background knowledge that related to our project with a brief introduction to details, including the fundamental of Enigma’s mechanism, brief introduction to classical cipher cryptanalysis, related modern machine learning techniques, and some of the on-going research in the machine learning and cryptanalysis area. At the end of this part, we made a summary of previous works and decided the direction of our research and implementation. Chapter 2 is an in-depth review of our implementation of experiment. Include the toolkits and APIs we used to generate training data, the building and optimising our networks, the details of the architecture and functional of our networks and data generator, and the metric we used to evaluate our networks. Chapter 3 shows the results of our experiments. Include the performance of networks with different hyperparameters settings, the learning curve of networks, and some of the attempts we made on flexibility on sequence length and harder attack with higher complexity. In chapter 4, we draw conclusions based on the results from experiments. We also discuss briefly about the possibility of applying machine learning on the cryptanalysis of modern ciphers and some of the improvement we can made in the future. 

This report contains four chapters:
\begin{itemize}
  \item  Chapter 1, the current chapter, provides a brief overview of the aims and objectives of this project.
  \item  Chapter 2 is the Literature review of our project. In this chapter, we reviewed the background knowledge that related to our project with a brief introduction to details, including the fundamental of Enigma’s mechanism, brief introduction to classical cipher cryptanalysis, related modern machine learning techniques, and some of the on-going research in the machine learning and cryptanalysis area. At the end of this chapter, we made a summary of previous works and decided the direction of our research and implementation.
  \item  Chapter 3 gives an in-depth review of our implementation of experiment. Include the toolkits and APIs we used to generate training data, the building and optimising our networks, the details of the architecture and functional of our networks and data generator, and the metric we used to evaluate our networks.
  \item  Chapter 4 shows the results of our experiments. This include the performance of networks with different hyperparameters settings, the learning curve of networks, and some of the attempts we made on flexibility on sequence length together with harder attack with higher complexity.
  \item  In Chapter 5, we draw conclusions based on the results from experiments. We also discuss briefly about the possibility of applying machine learning on the cryptanalysis of modern ciphers and some of the improvement we can made in the future. 